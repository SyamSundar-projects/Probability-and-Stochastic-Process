

\let\negmedspace\undefined
\let\negthickspace\undefined
\documentclass[journal,12pt,twocolumn]{IEEEtran}
%\documentclass[conference]{IEEEtran}
%\IEEEoverridecommandlockouts
% The preceding line is only needed to identify funding in the first footnote. If that is unneeded, please comment it out.
\usepackage{cite}
\usepackage{graphicx}
\usepackage{amsmath,amssymb,amsfonts,amsthm}
\usepackage{amsmath}
\usepackage{algorithmic}
\usepackage{graphicx}
\usepackage{textcomp}
\usepackage{xcolor}
\usepackage{txfonts}
\usepackage{listings}
\usepackage{enumitem}
\usepackage{mathtools}
\usepackage{gensymb}
\usepackage[breaklinks=true]{hyperref}
\usepackage{tkz-euclide} % loads  TikZ and tkz-base
\usepackage{listings}
\usepackage{placeins}
\usepackage{caption}

\graphicspath{ {./images/} }

\DeclareMathOperator*{\Res}{Res}
\renewcommand\thesection{\arabic{section}}
\renewcommand\thesubsection{\thesection.\arabic{subsection}}
\renewcommand\thesubsubsection{\thesubsection.\arabic{subsubsection}}

\renewcommand\thesectiondis{\arabic{section}}
\renewcommand\thesubsectiondis{\thesectiondis.\arabic{subsection}}
\renewcommand\thesubsubsectiondis{\thesubsectiondis.\arabic{subsubsection}}

% correct bad hyphenation here
\hyphenation{op-tical net-works semi-conduc-tor}
\def\inputGnumericTable{}                                 %%

\lstset{
%language=C,
frame=single, 
breaklines=true,
columns=fullflexible
}


\begin{document}
%


\newtheorem{theorem}{Theorem}[section]
\newtheorem{problem}{Problem}
\newtheorem{proposition}{Proposition}[section]
\newtheorem{lemma}{Lemma}[section]
\newtheorem{corollary}[theorem]{Corollary}
\newtheorem{example}{Example}[section]
\newtheorem{definition}[problem]{Definition}

\newcommand{\BEQA}{\begin{eqnarray}}
\newcommand{\EEQA}{\end{eqnarray}}
\newcommand{\define}{\stackrel{\triangle}{=}}
\bibliographystyle{IEEEtran}

\providecommand{\mbf}{\mathbf}
\providecommand{\pr}[1]{\ensuremath{\Pr\left(#1\right)}}
\providecommand{\qfunc}[1]{\ensuremath{Q\left(#1\right)}}
\providecommand{\sbrak}[1]{\ensuremath{{}\left[#1\right]}}
\providecommand{\lsbrak}[1]{\ensuremath{{}\left[#1\right.}}
\providecommand{\rsbrak}[1]{\ensuremath{{}\left.#1\right]}}
\providecommand{\brak}[1]{\ensuremath{\left(#1\right)}}
\providecommand{\lbrak}[1]{\ensuremath{\left(#1\right.}}
\providecommand{\rbrak}[1]{\ensuremath{\left.#1\right)}}
\providecommand{\cbrak}[1]{\ensuremath{\left\{#1\right\}}}
\providecommand{\lcbrak}[1]{\ensuremath{\left\{#1\right.}}
\providecommand{\rcbrak}[1]{\ensuremath{\left.#1\right\}}}
\theoremstyle{remark}
\newtheorem{rem}{Remark}
\newcommand{\sgn}{\mathop{\mathrm{sgn}}}
\providecommand{\abs}[1]{\left\vert#1\right\vert}
\providecommand{\res}[1]{\Res\displaylimits_{#1}} 
\providecommand{\norm}[1]{\left\lVert#1\right\rVert}

\providecommand{\mtx}[1]{\mathbf{#1}}
\providecommand{\mean}[1]{E\left[ #1 \right]}
\providecommand{\fourier}{\overset{\mathcal{F}}{ \rightleftharpoons}}

\providecommand{\system}{\overset{\mathcal{H}}{ \longleftrightarrow}}
	
\newcommand{\solution}{\noindent \textbf{Solution: }}
\newcommand{\cosec}{\,\text{cosec}\,}
\providecommand{\dec}[2]{\ensuremath{\overset{#1}{\underset{#2}{\gtrless}}}}
\newcommand{\myvec}[1]{\ensuremath{\begin{pmatrix}#1\end{pmatrix}}}
\newcommand{\mydet}[1]{\ensuremath{\begin{vmatrix}#1\end{vmatrix}}}

\let\vec\mathbf

\vspace{3cm}
\title{

Pseudo Sequence Noise Generator on Bread Board

}
\author{ Syam SundarMalla(AI23MTECH02002)$^{}$
}	

\maketitle
\newpage

\bigskip
\renewcommand{\thefigure}{\theenumi}
\renewcommand{\thetable}{\theenumi}

{\textbf{Abstract:}} A pseudo-random sequence noise generator, such as a Linear Feedback Shift Register (LFSR), is commonly used to generate pseudo-random sequences.In PN-Sequence generation also use same concept LFSR.\\

\section{Implementation}
A pseudo sequence noise (PN) generator can also be implemented using logic gates.Uses shift registers and exclusive OR (XOR) gates.

\begin{figure}[htbp]
    \centering
    \includegraphics[width=0.5\textwidth]{circuit_diag.png}
    \captionsetup{justification=centering}
    \caption{Circuit Diagram}
\end{figure}
  
By the Fig0 can observe that for four bit psedu noise sequence generator requires four registers and one XOR.\\ 
\\
Let us see how the random numbers are generating.\\

\begin{figure}[htbp]
    \centering
    \includegraphics[width=0.4\textwidth]{Truth_table.png}
    \captionsetup{justification=centering}
    \caption{Truth Table}
\end{figure}

Based on the seed(intial value after reset) it generates a new sequence of numbers.All the all zero state is not allowed.
For example if seed is 0101.The circuit will generate the following sequence in the truth table above.

\section{Applications}
1.PN Sequence generation which used as noise in Digital Dommunication.\\
2.Random numbers generation to allocate resources in Computer Organization\\

\section{ICs Required for Hardware Implementation}
a.555 Timer for clock generation \\
b.7486 for XOR operatioin\\
c.Two 7474 for four flip flops\\
d.7447 for BCD to 7 segment display decoder \\
e.seven segment display\\
f.Two 100nF capacitors\\
g.16MZ resistor\\

\end{document}

modify the above latex code for report with good indendations